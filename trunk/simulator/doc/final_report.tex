\documentclass[12pt,a4paper]{article}
\usepackage{fontspec} %字体包
\usepackage{xunicode} %unicode
\usepackage{indentfirst} %缩进包
\usepackage{graphicx} %图片包
\usepackage{color} %颜色包
\usepackage{fancyhdr} %页眉页脚
\usepackage{listings}
\usepackage{times}
%\usepackage{tikz}
\lstset{language=C++,tabsize=4, keepspaces=true,
    backgroundcolor=\color{white}, % 
    frame=none,   % frame=single,表示要边框; trbl
    keywordstyle=\color{blue}\bfseries,
    breakindent=22pt,
    numbers=left,stepnumber=1,
    basicstyle=\footnotesize,
    showspaces=false,
    flexiblecolumns=true,
    breaklines=true, breakautoindent=true,breakindent=4em,
    escapeinside={/*@}{@*/}
}

%\lstset{language=C++}%这条命令可以让LaTeX排版时将C++键字突出显示
%\lstset{breaklines}%这条命令可以让LaTeX自动将长的代码行换行排版
%\lstset{extendedchars=false}%这一条命令可以解决代码跨页时,章节标题,页眉等汉字不显示的问题

\pagestyle{fancy}
\lhead{\bfseries unicore32子集模拟器期末报告} 
\chead{} 
\rhead{\bfseries} 


\newcommand\song{\fontspec{宋体}}
\newcommand\hei{\fontspec{黑体}}
\newcommand\kai{\fontspec{楷体_GB2312}}
\newcommand\lishu{\fontspec{隶书}}
\newcommand\yahei{\fontspec{微软雅黑}}
%\newcommand\lishu{\fontspec[ExternalLocation=/media/hda5/windows/Fonts/]{simli.ttf}}
%\setromanfont[BoldFont={"[simhei.ttf]"}]{"[simsun.ttc]"}
\setromanfont[BoldFont={黑体}]{楷体_GB2312}

\begin{document}
\XeTeXlinebreaklocale "zh"
\XeTeXlinebreakskip = 0pt plus 1pt

\author{\song{李春奇} \and \song{彭焯} \and \song{华连盛} \and \song{王衎}}
\title{算法设计与分析第一次作业}
\maketitle
\newpage
\section{项目概述}
本项目实现了一个unicore32体系结构的模拟器,模拟器实现了CPU的五级流水、分立的i-cache和d-cache、内存管理、动态指令统计等模块,实现的指令集是unicore32体系结构指令集的一个子集,与同组的四个人所做的编译器实现的c语言子集相对应。最终实现的目标是能够使得实验室提供的unicore-gcc编译得到的ELF可执行文件能够正确在模拟器上运行,同组的编译器能够正确的在模拟器上运行。

本项目由于和编译器同时展开,所以同组共有四名同学。在编译器和模拟器的分工上,有着一定的偏重,其中李春奇、彭焯同学偏重于模拟器的实现,华连盛、王\song{衎}同学偏重于编译器的实现。但是在设计阶段和后期编译目标代码生成、编译优化、模拟器综合验证部分是由四名同学一起来完成的,即李春奇、彭焯同学同时也参与了编译器方面的工作,华连盛、王\song{衎}同学也同时参与了模拟器的验证工作。

项目最终实现的模拟器能够在linux环境下正确运行,需要外部libelf库,请在项目svn的lib文件夹下解压安装。项目svn地址:https://minic.googlecode.com/svn/trunk/simulator

\section{模块说明}




\end{document}










