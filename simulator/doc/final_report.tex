\documentclass[12pt,a4paper]{article}
\usepackage{fontspec} %字体包
\usepackage{xunicode} %unicode
\usepackage{indentfirst} %缩进包
\usepackage{graphicx} %图片包
\usepackage{color} %颜色包
\usepackage{fancyhdr} %页眉页脚
\usepackage{listings}
\usepackage{times}
%\usepackage{tikz}

\lstset{language=C++,tabsize=4, keepspaces=true,
    backgroundcolor=\color{white}, % 
    frame=none,   % frame=single,表示要边框; trbl
    keywordstyle=\color{blue}\bfseries,
    breakindent=22pt,
    numbers=left,stepnumber=1,
    basicstyle=\footnotesize,
    showspaces=false,
    flexiblecolumns=true,
    breaklines=true, breakautoindent=true,breakindent=4em,
    escapeinside={/*@}{@*/}
}

%\lstset{language=C++}%这条命令可以让LaTeX排版时将C++键字突出显示
%\lstset{breaklines}%这条命令可以让LaTeX自动将长的代码行换行排版
%\lstset{extendedchars=false}%这一条命令可以解决代码跨页时,章节标题,页眉等汉字不显示的问题

\pagestyle{fancy}
\lhead{\bfseries unicore32子集模拟器期末报告} 
\chead{} 
\rhead{\bfseries} 


\newcommand\song{\fontspec{宋体}}
\newcommand\hei{\fontspec{黑体}}
\newcommand\kai{\fontspec{楷体_GB2312}}
\newcommand\lishu{\fontspec{隶书}}
\newcommand\yahei{\fontspec{微软雅黑}}
%\newcommand\lishu{\fontspec[ExternalLocation=/media/hda5/windows/Fonts/]{simli.ttf}}
%\setromanfont[BoldFont={"[simhei.ttf]"}]{"[simsun.ttc]"}
\setromanfont[BoldFont={黑体}]{楷体_GB2312}

\begin{document}
\XeTeXlinebreaklocale "zh"
\XeTeXlinebreakskip = 0pt plus 1pt

\author{\song{李春奇} \and \song{彭焯} \and \song{华连盛} \and \song{王衎}}
\title{算法设计与分析第一次作业}
\maketitle
\newpage

\tableofcontents
\newpage

\section{实验目的和要求}
本项目实现了一个unicore32体系结构的模拟器,模拟器实现了CPU的五级流水、分立的i-cache和d-cache、内存管理、动态指令统计等模块,实现的指令集是unicore32体系结构指令集的一个子集,与同组的四个人所做的编译器实现的c语言子集相对应。最终实现的目标是能够使得实验室提供的unicore-gcc编译得到的ELF可执行文件能够正确在模拟器上运行,同组的编译器能够正确的在模拟器上运行。

\section{实验小组成员及分工}
本项目由于和编译器同时展开,所以同组共有四名同学。在编译器和模拟器的分工上,有着一定的偏重,其中李春奇、彭焯同学偏重于模拟器的实现,华连盛、王\song{衎}\kai{同}学偏重于编译器的实现。但是在设计阶段和后期编译目标代码生成、编译优化、模拟器综合验证部分是由四名同学一起来完成的,即李春奇、彭焯同学同时也参与了编译器方面的工作,华连盛、王\song{衎}\kai{同}学也同时参与了模拟器的验证工作。

在模拟器方面,彭焯同学负责cache模块的实现,李春奇同学负责其他模块的实现,华连盛、王\song{衎}\kai{同}学负责模拟器的验证工作。

\section{实验环境}
项目最终实现的模拟器能够在任何内核版本高于2.6.32的linux环境以及版本高于4.4.3的libc库环境下正确运行,需要外部libelf库,请在项目svn的lib文件夹下解压安装。

项目svn地址:https://minic.googlecode.com/svn/trunk/simulator

\section{模块说明}
\subsection{模块整体视图}
模块的视图如下:

\noindent\includegraphics[width=14cm]{module_view.eps}

通过模块视图可以看到,处于第零层的模块是Console模块,即所有模块最终被封装到了一个控制台上,这样Simulator解析命令行参数之后只需要通过调用控制台即可完成下面所有的调用工作。处于第一层的模块有CPU模块和Debugger模块,他们分别被Console Module调用,完成各自的功能。CPU模块是模拟CPU的模块,其子模块分为Process模块和Pipline模块,分别是模拟进程和流水线的模块。Process模块负责ELF文件的解析和对进程内存的初始化工作,所以起子模块是ELF Parser模块和Memory模块。Pipline模块则是模拟了CPU的五级流水,在五级流水中包括Cache模块、Register模块、Decode模块和指令执行模块。第一层的模块中另外一部分是Debugger模块,这一部分主要负责调试的工作,包含的子模块有Interpret模块(负责将机器码反汇编成汇编语言指令)和Breakpoint模块(设置和维护断点)。

以上就是项目的模块试图,每个模块对应项目中的一个文件,整体结构清晰明了,模块内部紧凑,模块之间相互独立,模块功能实现正交,便于实现和维护。下面就针对一些重要的模块(组)来进行说明,虽然在实现的过程中为了便于测试我们是采用自底向上的方式来实现,但是在说明的时候我们采用自顶向下的方式来说明,以便更清晰的说明层次关系。

对于项目整体设计的具体说明,请参考“Simulator设计文档.txt”。

\subsection{CPU模块}
CPU模块主要包括以下两个子模块:
\subsection{Process子模块}
该模块负责ELF文件的载入以及进程对应内存模块的装载。

ELF文件的解析是从装载者的角度来处理,主要载入几个需要载入的segments,并且需要载入符号表以及每个表项在字符串表中对应的字符串,以便在后面反汇编时应用。此外还需要解析main函数的入口地址,这个可以简单的在符号表中查找即可。ELF文件解析模块应用到了libelf库解析ELF文件。

Process子模块再解析完ELF文件之后需要建立该进程对应的内存,这里采用的是把整个进程都调入内存的方式,即没有设计页式内存管理机制,对于cache来说内存是平坦的。每个进程第一个segment为该进程的栈,栈底地址为0x10000000,栈向低地址方向生长,栈大小可以在编译前配置。从编号为1的segment开始为ELF文件中需要载入的段以此存储。在Memory模块中需要特别说明的是实现了内存映射的机制,即对于任何进程(simulator中的进程)来说,是看不到物理机的实际内存地址的,对于任何虚拟地址的访问,都会通过一个中间层来处理,中间层首先检查内存的合法性,在内存合法的前提下进行映射转换,得到对应的数据。这样的设计保证了模拟器上运行的进程对内存地址访问的合法性检查,如果在模拟器运行阶段发现跑在上面的进程访问了不正确的内存,即报告模拟器的错误Invalid address,如果是产生了系统的segment fault,那么则是模拟器的实现有问题。这样可以有效的区分模拟器的错误和编译器的错误,在二者联合调试的时候这一点设计为我们节约和很多调试的时间。

\subsection{Pipline子模块}
Pipline模块是模拟器的核心,该模块模拟了CPU的五级流水,并且在流水中涉及到了对cache和register的操作。在说明五级流水之前,首先说明一下cache和register莫口爱

\subsubsection{Cache子模块}
模拟器采用的cache是单片cache、I-cache和D-cache分离的Havard结构。对于CPU来说是不能够直接看到进程的内存的(在这里是被虚拟化和映射过的内存),CPU只能直接对cache操作,而cache负责处理与内存间的数据一致性等问题。cache预设的未命中等待的CPU周期为8个周期,


\end{document}










