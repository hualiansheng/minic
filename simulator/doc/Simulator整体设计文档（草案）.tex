\documentclass[12pt,a4paper]{article}
\usepackage{fontspec} %字体包
\usepackage{xunicode} %unicode
\usepackage{indentfirst} %缩进包
\usepackage{graphicx} %图片包
\usepackage{color} %颜色包
\usepackage{fancyhdr} %页眉页脚
\usepackage{listings}
\usepackage{times}
%\usepackage{tikz}

\lstset{language=C++,tabsize=4, keepspaces=true,
    backgroundcolor=\color{white}, % 
    frame=none,   % frame=single,表示要边框; trbl
    keywordstyle=\color{blue}\bfseries,
    breakindent=22pt,
    numbers=left,stepnumber=1,
    basicstyle=\footnotesize,
    showspaces=false,
    flexiblecolumns=true,
    breaklines=true, breakautoindent=true,breakindent=4em,
    escapeinside={/*@}{@*/}
}

%\lstset{language=C++}%这条命令可以让LaTeX排版时将C++键字突出显示
%\lstset{breaklines}%这条命令可以让LaTeX自动将长的代码行换行排版
%\lstset{extendedchars=false}%这一条命令可以解决代码跨页时,章节标题,页眉等汉字不显示的问题

\pagestyle{fancy}
\lhead{\bfseries unicore32子集模拟器期末报告} 
\chead{} 
\rhead{\bfseries} 


\newcommand\song{\fontspec{宋体}}
\newcommand\hei{\fontspec{黑体}}
\newcommand\kai{\fontspec{楷体_GB2312}}
\newcommand\lishu{\fontspec{隶书}}
\newcommand\yahei{\fontspec{微软雅黑}}
%\newcommand\lishu{\fontspec[ExternalLocation=/media/hda5/windows/Fonts/]{simli.ttf}}
%\setromanfont[BoldFont={"[simhei.ttf]"}]{"[simsun.ttc]"}
\setromanfont[BoldFont={黑体}]{楷体_GB2312}

\begin{document}
\XeTeXlinebreaklocale "zh"
\XeTeXlinebreakskip = 0pt plus 1pt

\author{\song{李春奇} \and \song{彭焯} \and \song{华连盛} \and \song{王衎}}
\title{Unicore32模拟器整体设计文档(草案)}
\maketitle
模拟器架构主要分为三个部分:进程模块、CPU模块和调试模块
\section{进程模块}
	进程模块的接口声明在文件process.h中,实现在process.c中;进程模块包含的子模块有:内存管理模块、ELF文件解析模块。

	进程模块的功能描述:进程模块的功能是建立一个进程,包括进程自己内存空间的初始化和维护、进程代码和数据的载入、程序入口点等。

	进程模块的接口如下:
	
\begin{lstlisting}[language={C}]
	extern PROCESS* proc_initial(char* filename);
    extern int proc_destroy(PROCESS* proc);
\end{lstlisting}
	
    功能分别为建立进程和销毁进程,其中filename为进程的物理文件名(ELF可执行文件),proc为需要销毁的进程。

\subsection{内存管理子模块(memory.h memory.c)}
\begin{enumerate}
    \item 内存管理子模块分为两个部分:栈空间管理和内存空间管理
    \item 栈空间管理
    \item 内存空间管理
\end{enumerate}

\subsection{ELF文件解析模块(ELF\_parser.h ELF\_parser.c)}
	ELF文件解析模块主要是用来解析ELF文件,建立对应的Segment,并将栈作为Segment统一管理。

\section{CPU模块}
  CPU模块的接口声明在文件CPU.h中,实现在CPU.c中;CPU模块包含的子模块有:寄存器堆模块、流水线模块、cache模块

\subsection{寄存器堆模块}
	寄存器堆模块维护的是每个CPU对应的寄存器堆,主要需要维护CMSR的N、Z、C、V的读和写。
	
\subsection{cache模块}
	Cache模块需要实现如下功能:
	\begin{enumerate}
		\item Cache的读写
		\item Cache与内存的一致性问题,即对于写脏的内存的Cache更新策略。
		\item Cache的Miss和Hit机制
		\item Cache的写回机制
	\end{enumerate}
	
\subsection{流水线设计}	
流水线设计是模拟器设计的重点,下面对流水线进行详细的描述。

\subsubsection{流水结构}
五级流水:

取指(IF)--译码(读寄存器)(ID)--ALU计算(EX)--访存(MEM)--写回(WB)

\subsubsection{需要解决的冒险}
\begin{enumerate}
\item 结构冒险:通过指令cache和数据cache分离的Havard结构解决
\item 数据冒险:
	\subitem 数据转发机制解决一部分冒险问题
	\subitem 但是“加载-使用型数据冒险”需要加一个气泡
\item 控制冒险:
	\subitem 分支控制导致的冒险:用分支预测的方法来解决:
	\subitem 分支预测的策略:a.B指令一定预测跳转,b.其他条件跳转指令预测不跳转
\end{enumerate}

\subsection{各级流水的设计和接口规范}
\subsubsection{IF(Instruction Fetch)}
输入:指令地址

输出:struct

实现:直接对cache进行访问,分为命中和未命中两种情况


\subsubsection{ID(Instruction Decode)}
输入:IF的输出

输出:struct

实现:对指令进行译码,获得对应寄存器的值,寄存器的编号,操作类型、操作数、移位立即数的值等等并保存在结构体中

\subsubsection{Ex(Execuation)}
输入:ID的输出

输出:struct

实现:根据指令的类型进行相应的运算,对于R型指令,写回在此处进行(即在此处进行数据转发);对于分支指令,分支预测在此处进行处理。

\subsubsection{Mem(Memory)}
输入:Ex的输出

输出:struct

实现:lw指令的写回在此处实现,其他指令到此处均已执行完毕

\subsubsection{WB(Write Back)}
所有指令均已经执行完毕,此模块只做一些初始化工作。

\subsection{各模块调用次序及相关结构的定义}
各模块在逻辑上是并行的,在实现上是从后向前调用,遇到需要插入气泡的情况直接阻塞前边的各级流水

上文所述的结构:
	
\begin{lstlisting}[language={C}]
typedef struct{
    uint32_t inst_addr;
    uint32_t inst_code;
    int inst_type;
    int opcodes;
    uint32_t Rn, Rd, Rs, Rm;
    int imm;
    int shift;
    int rotate;
    int cond;
    int high_offset, low_offset;
    int S, A, P, U, B, W, L, H;
    uint32_t cur_inst_PC;
    uint32_t addr;
}PIPLINE_DATA;
\end{lstlisting}

流水线的数据结构如下:

\begin{lstlisting}[language={C}]
typedef struct{
    int block;//1 means pipline block, 0 mean the opposite
    int block_reg;
    PIPLINE_DATA* pipline_data[PIPLINE_LEVEL];
    char pipline_info[PIPLINE_LEVEL][200];
    PROC_STACK* stack;
    REGISTERS* regs;
    CACHE *i_cache, *d_cache;
    PROCESS* proc;
    int drain_pipline;
    int pc_src;
    int ex_begin;
}PIPLINE;
\end{lstlisting}


\end{document}










