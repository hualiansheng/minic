%CTeX的report文档类   如果是article就用ctexart
\documentclass[a4paper,winfonts]{ctexrep}
% ------------------------------------------------------------------------------------------------------
%额外的数学符号支持
%\usepackage{amsthm}
%额外的注脚支持
\usepackage[stable]{footmisc}
%caption for environment
\usepackage{caption}
%页边距设置
\usepackage[top=0.8in,bottom=0.8in, left=0.8in, right=0.8in]{geometry}
%图片引用及格式支持
\usepackage{graphicx}
%长表格支持
\usepackage{longtable}
%more symbols
\usepackage{dingbat}
\usepackage{manfnt}
%超链接目录支持
\usepackage[CJKbookmarks=true,colorlinks,linkcolor=blue]{hyperref}    
%彩色支持、更多的色彩名称
\usepackage[usenames,dvipsnames]{color}
%代码高亮段支持
\usepackage{listings} %source code printing support
%subsubsubsection支持
\setcounter{secnumdepth}{5}
\renewcommand\theparagraph{\alph{paragraph}} \usepackage{lipsum}

%浅灰色(代码底色)
\definecolor{lightgray}{RGB}{230,230,230}
%常用emacs命令
%C-c C-q C-s(-e) Indentation
%C-c C-s new section
%C-c C-e new environment
%C-c C-c compile
%C-c C-v preview
\setmonofont{Courier New}
\lstset{ language={[ANSI]C},
         showspaces=false,
         showtabs=false,
         tabsize=4,
         frame=single,
         framerule=1pt,
         %numbers=left,
         %numberstyle=\small,
         basicstyle=\tt,
         directivestyle=\tt,
         identifierstyle=\tt,
         commentstyle=\tt,
         stringstyle=\tt,
         keywordstyle=\color{blue}\tt }
\usepackage{dirtree}
