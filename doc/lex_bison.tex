\section{词法分析, flex}
\label{flex}
词法分析的目的是将输入的源文件中的符号识别为语法分析器能够接受的tokens。其基本原理是利用正则表达式pattern来匹配、分割源文件中的符号。同时,词法分析还需要识别并保存一些名字和常量,比如函数名、变量名和字符串常量。

MiniC的词法规则同标准C的对应规则基本相同:
\begin{enumerate}
	\item 标识符必须以下划线或大小写字母开头,由下划线、大小写字母和数字组成
	\item 字符常量需要放在单引号\verb|''|中,字符串常量需要放在双引号\verb|""|中
	\item 有如下保留字不能当作标识符:\verb|extern|, \verb|register|, \verb|void|, \verb|int|, \verb|char|, \verb|if|, \verb|else|, \verb|for|, \verb|while|, \verb|return|
	\item 特殊符号包括:\verb#{}, (), [], +, -, *, !, &, =, |, >, <#
\end{enumerate}
由于语法分析器bison实际上不能获得\hyperref[ASTnode]{AST叶节点}上的信息,因此生成AST叶节点的工作交由flex完成。\\

{\it \anchor flex源文件请参阅:\verb|minic.l|}
\section{语法分析, bison}
\label{bison}
语法分析的目的是根据语言的BNF范式,将词法分析器提交的token流进行规约,在规约的同时应用一些语法规则。语法分析的结果有两个:
\begin{enumerate}
	\item 将\emph{语法正确}的源文件转换为语法规则所要求的中间形式,这一中间形式不在具有语言的语法特性,或
	\item 发现语法错误,如果错误能够暂时恢复,就继续语法分析,在完成后提示错误信息;否则,直接停止语法分析,报告错误。
\end{enumerate}
MiniC项目利用bison辅助进行语法分析,分析完成后,将建立一棵AST。

在介绍对bison的利用前,由于我们的项目修改了给定的MiniC文法,所以首先要对修改后的文法进行说明。
\subsection{MiniC文法}
我们实现的MiniC文法和实习项目开始时给定的有所不同,最大的不同之处在表达式文法部分。
以下是原表达式文法:
\begin{verbatim}
unary_expr := unary_op unary_expr | postfix_expr
unary_op := ! | + | - | ++ | -- | & | *
postfix_expr := postfix_expr ( expression )
	| ident ( argumentlist )
	| ident ( )
	| postfix_expr ++
	| postfix_expr --
	| primary_expr
\end{verbatim}
注意到这种文法允许诸如下述的语法:
\begin{itemize}
	\item \verb|++ ++ a;|(等价于\verb|a=a+1=a+1|)
	\item \verb|!a = 1;|
	\item \verb|f()=1;|
\end{itemize}
而这些语法实际上是语言所不能处理的,因为“计算结果”和“代表一个存储位置的符号”是完全不同的概念,前者没有地方存储也因此不能被赋值。

我们修改后的表达式文法基于“左值”——代表一个存储区域的符号和“右值”——计算结果,并且是二义性文法:
\begin{verbatim}
expression := rvalue | assignment_expression
assignment_expression:= lvalue = assignment_expression 
	| lvalue = rvalue
lvalue := * rvalue | IDENT | IDENT [ expression ]
rvalue := lvalue
	| rvalue + rvalue
	| rvalue - rvalue
	| rvalue * rvalue
	| rvalue op rvalue
	| ( rvalue )
	| + rvalue
	| - rvalue
	| ! rvalue
	| & lvalue
	| IDENT ( argument_list )
	| IDENT ( )
	...
\end{verbatim}
这样定义的文法在表达的意义上更清晰,同时也避免了上述的错误。\\
{\it \anchor 有关原文法漏洞的详细说明,请参阅:\verb|BNF_leaks.pdf|}\\
{\it \anchor 本项目的全部BNF范式请参阅:\verb|BNF.pdf|}\\

我们对原MiniC文法的改动主要在其表达式文法部分。

\subsection{利用bison进行文法分析}
bison是一个能够自动构建$LALR(1)$(或$LR(K)$)语法分析器的软件。它的输入是一个文法描述,输出是一个能够直接编译的C源文件,只需调用\verb|yyparse()|函数,就可以开始按照文法进行规约,同时执行相应的文法规则。

使用bison需要注意几个问题:
\begin{enumerate}
\item LR分析允许文法中有左递归
\item bison允许使用二义性文法,但是需要指名一个规约规则,否则会提示规约-规约冲突。例如,为了使文法设计简单,MiniC使用的表达式文法就是二义性文法,但是利用bison提供的“定义规约优先级”和“定义结合律”功能,就可以实现按照指定的算符优先级来进行规约。
\item 每个终结符和非终结符都有同样的“属性”结构,因此,在声明这个结构时需要考虑到属性文法中所要用到的所有属性。下面是MiniC语法分析器所用的属性结构:
	\begin{lstlisting}
%union{
	iattr int_attr; //可能具有的整数属性
	cattr char_attr; //可能具有的字符属性
	sattr string_attr; //可能具有的字符串属性
	AST_NODE* ptr; //指向子节点的指针
}
	\end{lstlisting}
只有叶子节点才会使用上面三种属性,因为它们可能包含标识符名称或常量。

\end{enumerate}
{\it \anchor 本项目的bison源文件请参阅:\verb|minic.y|}\\
\subsection{语法错误检查}

\label{syntexerror}
bison提供了完善的错误检查功能,当发现一个语法错误时,它会暂停当前的规约过程,并在当前规约到的表达式后添加一个"\verb|error|"终结符,然后开始在规约栈中从\verb|error|以下弹出符号,直到找到一个带有"\verb|error|"终结符的产生式,便用它规约,并应用相应的文法规则(一般来说是错误处理代码);然后文法分析恢复正常运行。如果找不到用于处理错误的文法,那么分析器报错退出。

与此同时,bison还提供了追踪终结符/非终结符在源文件中的位置的机制,只需要词法分析器在一个名为\verb|yylloc|的结构中填入信息,这个结构中记录了终结符的开始行列编号和结束行列编号。bison将在规约的过程中自动维护每个文法符号的\verb|yylloc|。

MiniC添加了简单的错误处理/错误代码追踪功能。我们在\verb|lvalue|和\verb|rvalue|产生式集中各添加了一个\verb|error|单一产生式,并且加入了报错的代码,因此,一旦MiniC源文件中表达式出现错误,语法分析器就会指示其位置。

例如,下面代码第4行少了一个分号:
\begin{lstlisting}
int main()
{
	int i;
	i = i + 1
}
\end{lstlisting}
编译器会提示:
\begin{verbatim}
Syntax error
Parsing error: line 4.0-5.1: bad lvalue
 Parsing error: line 4.0-5.1: bad lvalue
 Parser: terminated, 2 error(s).
\end{verbatim}
\section{谬误与陷阱}
\label{pitfallc1}
\paragraph*{谬误1:}文法的设计可以草率为之,将问题留给语义检查会比考虑一个正确的文法更方便。

文法具有其数学上的严谨性,一条有漏洞的文法可能会导致允许的语言集合过大或过小。最为危险的是如果这个集合过大,那么一类或几类错误的语言集合就会顺利通过语法分析:这类集合在文法上应当具有相似的结构,但是在语义上就可能很不相同,如果在语义分析时再检查这些错误,将造成复杂的代码逻辑和冗长的分支。

\paragraph*{谬误2:}语法所支持的就是语言应支持的。

通过了语法分析仅仅以为着符合语法的规范,确不意味着语义的正确。例如:\verb|void|关键字可以修饰一个变量,这是符合MiniC语法的,然而,一个\verb|void|形的变量既不能赋值又不能引用,如果允许它出现,那么在生成三元式,生成目标代码时将会不知道如何去处理它。所以我们发现了这个问题后,就MiniC的语义分析中,禁用了这种表示。

